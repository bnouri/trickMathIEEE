\RequirePackage[l2tabu, orthodox]{nag}
\documentclass[journal]{IEEEtran}
\usepackage{BNieee}
\begin{document}
\bstctlcite{IEEEexample:BSTcontrol} %For the controls in bib file 
%%%%%%%%%%%%%%%%%%%%%%%%
%%%%%%   Title:   %%%%%%
%%%%%%%%%%%%%%%%%%%%%%%%
\title{Math Equations in IEEEtran.sty}
\author{Behzad~Nouri}
\maketitle
\begin{abstract}
	In this document, I will use the following packages in preamble with the intention of keeping it as simple as possible.
	
	To restore IEEEtran’s ability to automatically break within multiline equations, load \texttt{amsmath} in
	preamble like:
	\begin{verbatim}
	\usepackage{amsmath}
	\interdisplaylinepenalty=2500
	\usepackage{amssymb}  %Inld: \mathbb{}
	\end{verbatim}
\verb!amsmath! is a comprehensive package which contains many helpful tools including enhancing the multiline alignment environments.	
\end{abstract}

\bigskip
\par\noindent \dotfill
\par\noindent \textbf{Note:} Marked by {\color{red} \Large $\mathbf{\times}$} shows it not a preferred solution with \texttt{IEEEtran.sty}!
\par\noindent\dotfill

%--------------------------------------------------------------------------
\tableofcontents
%--------------------------------------------------------------------------


\newpage
%--------------------------------------------------------------------------
\section{In-Line Equations \texttt{(\$ \$)}}
%--------------------------------------------------------------------------
In line equation $\frac{x}{y}$ as \verb!$\frac{x}{y}$!\par
Or in-line display: $\dfrac{x}{y}$ as \verb!$\dfrac{x}{y}$!



%--------------------------------------------------------------------------
\noindent \dotfill
\section{Display Equations \texttt{(\$\$ \$\$)}} \label{ch:display-eq}
%--------------------------------------------------------------------------
\[x = \sum_{i=0}^{z} 2^{i}Q\]

\verb!\[x = \sum_{i=0}^{z} 2^{i}Q\]!


%--------------------------------------------------------------------------
\noindent \dotfill
\section{Enumerated Equations}
%--------------------------------------------------------------------------
Equations are created using the traditional equation environment as

\begin{equation} 
a = \sum_{k=1}^n\sum_{\ell=1}^n \sin \bigl(2\pi \, b_k \, c_{\ell} \, h \bigr)
\end{equation}

\begin{verbatim}
\begin{equation}  \label{eq:}
a = \sum_{k=1}^n\sum_{\ell=1}^n
\sin \bigl(2\pi\,b_k\,c_{\ell}\,h\bigr)
\end{equation}
\end{verbatim}

%--------------------------------------------------------------------------
\noindent \dotfill
\section{Non-Enumerated Equations {\color{red} \Large $\mathbf{\times}$}}
%--------------------------------------------------------------------------
Another possibility to display an equation without showing its number is as shown below.

\begin{equation*}
x = \sum_{i=0}^{z} 2^{i}Q 
\end{equation*}

\begin{verbatim}
\begin{equation*}
x = \sum_{i=0}^{z} 2^{i}Q
\end{equation*}
\end{verbatim}

But it is preferable to use the \verb!displaymath! environment in \Sec{ch:display-eq} instead of the ``Non-Enumerated'' form.

If an equation is too big to be written in one line We also can split it in two or more lines as shown in the following sections.

%--------------------------------------------------------------------------
\noindent \dotfill
\section{Standard \texttt{eqnarray}: {\color{red} \Large $\mathbf{\times}$}}
%--------------------------------------------------------------------------
One way is using the standard (\LaTeX-built-in) \texttt{eqnarray} environment. \LaTeX\ built-in \verb!eqnarray! but it has  several shortcomings. Nevertheless, it is instructive to show a simple example using the standard \verb!eqnarray! in order to show some math spacing under \LaTeX. 

\setlength{\arraycolsep}{0.0em}
\begin{eqnarray}
Z&{}={}&x_1 + x_2 + x_3 + x_4 + x_5 + x_6\nonumber\\
&&{+}\:a + b
\end{eqnarray}
\setlength{\arraycolsep}{5pt}

\begin{verbatim}
\setlength{\arraycolsep}{0.0em}
\begin{eqnarray}
Z&{}={}&x_1 + x_2 + x_3 + x_4 + x_5 + x_6
\nonumber\\
&&{+}\:a + b
\end{eqnarray}
\setlength{\arraycolsep}{5pt}
\end{verbatim}

In \TeX’s math mode, spacing around operators can be inhibited by enclosing them within
braces (e.g., \verb!{=}\:!) or forced by surrounding them with “empty ords” (e.g., \verb!{}={}!).
Both are accepted. The empty  \verb!{}! do not have width themselves.

%--------------------------------------------------------------------------
\noindent \dotfill
\section{\texttt{multline} {\color{red} \Large $\mathbf{\times}$}}
%--------------------------------------------------------------------------
Using \verb!multline! Environment:

\begin{multline}
Z=x_1 + x_2 + x_3 + x_4 + x_5 + x_6 + x_7 + x_8 + x_9 + x_{10}\\
+ a + b
\end{multline}

\begin{verbatim}
\begin{multline} \label{eq: }
Z=x_1 + x_2 + x_3 + x_4 + x_5 + x_6 + x_7 +
x_8 + x_9 + x_{10}\\
+ a + b
\end{multline}
\end{verbatim}


%--------------------------------------------------------------------------
\noindent \dotfill
\section{\texttt{spilt}}
%--------------------------------------------------------------------------
Using \verb!spilt! environment is recommended for the typical IEEE style!

Like \verb!multline!, the \verb!split! environment is for single equations that are too long
to fit on one line. Unlike \verb!multline!, \verb!split! can use \& to mark alignment. 
Unlike the other amsmath equation structures, the split environment provides no numbering and must be used inside some others such as equation, align,...

\begin{equation}
\begin{split}
\begin{split}
H=& \sum_l\prod\limits^p_{i=1} 
\binom{n_i}{l _i} \\
& \times \sum^p_{i=1}n_i
\end{split}
\end{split}
\end{equation}

\begin{verbatim}
\begin{equation} \label{eq: }
\begin{split}
H=& \sum_l\prod\limits^p_{i=1} 
\binom{n_i}{l _i} \\
& \times \sum^p_{i=1}n_i
\end{split}
\end{equation}
\end{verbatim}

%--------------------------------------------------------------------------
\noindent \dotfill
\section{\texttt{align}}
%--------------------------------------------------------------------------
It is from \texttt{amsmath.sty} package!

\begin{align}
Z=&x_1 + x_2 + x_3 + x_4 + x_5 + x_6  \nonumber\\
&+ a + b
\end{align}

\begin{verbatim}
\begin{align}
Z=&x_1 + x_2 + x_3 + x_4 + x_5 + x_6  
\nonumber\\
&+ a + b
\end{align}
\end{verbatim}


It is trivial to see another uses for \verb!alighn! as:

\begin{align}
Z&=x_1+x_2+x_3+x_4+x_5+x_6\nonumber\\
&=a + b
\end{align}


\begin{verbatim}
\begin{align} \label{eq: }
Z&=x_1+x_2+x_3+x_4+x_5+x_6\nonumber\\
&=a + b
\end{align}
\end{verbatim}


\subsection{\texttt{align}: Multi-Col Single-Row}

\begin{align}
A=1 && B=2
\end{align}

\begin{verbatim}
\begin{align} \label{eq: }
A=1 && B=2
\end{align}
\end{verbatim}

\begin{align}
A=1 && B=2 && C=3 && D=4 && F=5
\end{align}

\begin{verbatim}
\begin{align} \label{eq: }
A=1 && B=2 && C=3 && D=4 && F=5
\end{align}
\end{verbatim}


\subsection{\texttt{align}: Multi-Col Multi-Row}

\begin{align}
A&=B  &C&=D\nonumber\\
E&=F  &G&=H
\end{align}

\begin{verbatim}
\begin{align}
A&=B  &C&=D\nonumber\\
E&=F  &G&=H
\end{align}
\end{verbatim}

and

\begin{align}
a&=b &  c&=d &  e&=f \nonumber\\
A&=B &  C&=D &  E&=F \nonumber\\
M&=N &  K&=k &  l&=i+j
\end{align}

\begin{verbatim}
\begin{align} \label{eq: }
a&=b &  c&=d &  e&=f \nonumber\\
A&=B &  C&=D &  E&=F \\
M&=N &  K&=k &  l&=i+j
\end{align}
\end{verbatim}

%-------------------------------------------------------------------------------------------
\subsection{BAD use for \texttt{align}~{{\color{red} \Large $\mathbf{\times}$}}}
%-------------------------------------------------------------------------------------------
\mbox{}\par \noindent Note that, in some cases:\\
{\color{red} with \verb!align! we can not have a beautiful fomatting}.\par
\textbf{e.g.:} see the separation in the \verb!align! to fill the column as shown below.\\

\begin{align}
\tilde{C} &= Q^TCQ,  &\hfill&  &\tilde{G}&=Q^TGQ \nonumber\\
\tilde{B} &= Q^TB,   &\hfill&  &\tilde{L}&=LQ.
\end{align}

\begin{verbatim}
\tilde{C} &= Q^TCQ,  &\hfill&  
&\tilde{G}&=Q^TGQ \nonumber\\
\tilde{B} &= Q^TB,   &\hfill&  
&\tilde{L}&=LQ.
\end{verbatim}

%--------------------------------------------------------------------------
\noindent \dotfill
\section{\texttt{alignat}}
%--------------------------------------------------------------------------
\textbullet~~In \verb!\begin{alignat}{lr}!, $lr$ is the number of equation columns. 
\[lr=(\text{max. num. of in any row } \& + 1)/2.\]
\centerline{e.g.:~~~~~~~\&~~~~col.1~~~~\&~~~~col.2~~~~\& }\\

\textbullet~~A variant environment \verb!alignat! allows the horizontal space between equations
to be explicitly specified in the equation.\\[8pt]

\begin{alignat}{2}
x&=y\dots         &\hspace{4pc}& \text{by (1)} \nonumber\\
&=y'\circ y^\ast &            & \text{by (2)}        \label{eq:xxxx2} \\
&=y(0)           &            & \text {by Axiom 1.}  \label{eq:xxxx3}
\end{alignat}

\begin{verbatim}
\begin{alignat}{2}
x&=y\dots   &\hspace{4pc}& \text{by (1)}
\nonumber\\
&=y'\circ y^\ast &   &\text{by (2)}
\label{eq:xxxx2} \\
&=y(0)           &   &\text{by Axiom-1.}
\label{eq:xxxx3}
\end{alignat}
\end{verbatim}

\begin{alignat}{3}
A&=B  \qquad  &C&=D \qquad &E&=F \nonumber\\
G&=H          &I&=J        &K&=M.
\end{alignat}


\begin{verbatim}
\begin{alignat}{3}\label{eq:}
A&=B  \qquad  &C&=D \qquad &E&=F \nonumber\\
G&=H          &I&=J        &K&=M.
\end{alignat}
\end{verbatim}

\begin{alignat}{3}
\tilde{C} &= Q^TCQ,  &\qquad&  &\tilde{G}&=Q^TGQ \nonumber\\
\tilde{B} &= Q^TB,   &      &  &\tilde{L}&=LQ.
\end{alignat}

\begin{verbatim}
\begin{alignat}{3}
\tilde{C} &= Q^TCQ, &\qquad& &\tilde{G}&=
Q^TGQ \nonumber\\
\tilde{B} &= Q^TB,  &      & &\tilde{L}&=
LQ. \lable{eq: } \end{alignat}
\end{verbatim}
%--------------------------------------------------------------------------

%--------------------------------------------------------------------------
\noindent \textbf{Examples:}\par
%--------------------------------------------------------------------------
\noindent Adjustable separation between two columns:
\begin{verbatim}
\begin{alignat}{2}
A=B  &\hspace{24pt}& M=N\nonumber\\
C=D  &             & P=Q \label{eq: }
\end{alignat}
\end{verbatim}

\begin{alignat}{2}
A=B  &\hspace{24pt}& M=N\nonumber\\
C=D  &             & P=Q
\end{alignat}

\begin{verbatim}
\mathat{C}&\trieq\mat{V}\T\mat{C}\mat{V}
\in\mathbb{R}^{m\times m} 
&\hspace{16pt}& 
&\mathat{G}&\trieq\mat{V}\T\mat{G}\mat{V}
\in\mathbb{R}^{m\times m} \label{eq: }\\
\mathat{B}&\trieq\mat{V}\T\mat{B}
\in\mathbb{R}^{m\times n_{in}} 
&             & 
&\mathat{L}&\trieq\mat{L}\mat{Q}
\in\mathbb{R}^{n_{out}\times m}\label{eq: }
\end{verbatim}

\begin{alignat}{3}
\mathat{C}&\trieq\mat{V}\T\mat{C}\mat{V}\in \mathbb{R}^{m\times m} &\hspace{16pt}& &\mathat{G}&\trieq\mat{V}\T\mat{G}\mat{V}\in \mathbb{R}^{m\times m} \\
\mathat{B}&\trieq\mat{V}\T\mat{B}\in \mathbb{R}^{m\times n_{in}} 
&             & 
&\mathat{L}&\trieq\mat{L}\mat{Q}\in \mathbb{R}^{n_{out}\times m}
\end{alignat}


%--------------------------------------------------------------------------
\noindent \dotfill
\subsection{\texttt{aligned}}
%--------------------------------------------------------------------------
\texttt{aligned} provides total width which is the actual width of the contents; thus they can be used as a component in a containing expression. These -ed variants also take an optional [t], [b]
or the default [c] argument to specify vertical positioning.

\begin{equation*}
\left.\begin{aligned}
B'&=-\partial\times E,\\
E'&=\partial\times B - 4\pi j,
\end{aligned}
\right\}
\qquad \text{Maxwell's equations}
\end{equation*}

\begin{verbatim}
\begin{equation*}
\left.\begin{aligned}
B'&=-\partial\times E,\\
E'&=\partial\times B - 4\pi j,
\end{aligned}
\right\}
\qquad \text{Maxwell's equations}
\end{equation*}
\end{verbatim}


%--------------------------------------------------------------------------
\noindent \dotfill
\subsection{\texttt{cases}}
%--------------------------------------------------------------------------
For multi-line equations in the \verb!amsmath! package there is a \verb!cases! environment to make them easy as
\begin{equation*}
\left|x\right|=
\begin{cases}
XX & \text{if} X=0, \\
YY & \text{if} YY\le 0
\end{cases}
\end{equation*}

\begin{verbatim}
\begin{equation*}
\left|x\right|=
\begin{cases}
XX & \text{if} X=0, \\
YY & \text{if} YY\le 0
\end{cases}
\end{equation*}
\end{verbatim}
%--------------------------------------------------------------------------
\noindent \dotfill
\subsection{\texttt{flalign}}
%--------------------------------------------------------------------------
The environment \verb!flalign! (``full length alignment'') stretches the space between the equation columns to the maximum possible width, leaving only enough space at the margin for the equation number.

\begin{flalign}
x&=y & X&=Y\\
x'&=y' & X'&=Y'\\
x+x'&=y+y' & X+X'&=Y+Y'
\end{flalign}

\begin{verbatim}
\begin{flalign}
x&=y & X&=Y\\
x'&=y' & X'&=Y'\\
x+x'&=y+y' & X+X'&=Y+Y'
\end{flalign}
\end{verbatim}

No equation number then no space:

\begin{flalign*}
x&=y & X&=Y\\
x'&=y' & X'&=Y'\\
x+x'&=y+y' & X+X'&=Y+Y'
\end{flalign*}

\begin{verbatim}
\begin{flalign*}
x&=y & X&=Y\\
x'&=y' & X'&=Y'\\
x+x'&=y+y' & X+X'&=Y+Y'
\end{flalign*}
\end{verbatim}

%--------------------------------------------------------------------------
\noindent \dotfill
\section{Interjection of text in the Equations} \label{ch:text-interjection}
%--------------------------------------------------------------------------

%--------------------------------------------------------------------------
\noindent \dotfill
\subsection{\texttt{flalign} for putting text in the equations}
%--------------------------------------------------------------------------

\begin{flalign}
&\text{L.H.S:}& PF(\alpha) &= A+B & \\
&\text{R.H.S:}& UC(\alpha) &= x & \nonumber \\
& & &= W & 
\end{flalign}

\begin{verbatim}
\begin{flalign}
&\text{L.H.S:}& PF(\alpha) &= A+B & 
\label{eq: }\\
&\text{R.H.S:}& UC(\alpha) &=x &\nonumber\\
& & &= W & \label{eq: }
\end{flalign}
\end{verbatim}

%--------------------------------------------------------------------------
\noindent \dotfill
\subsection{\texttt{align} for putting text in the multi-line}
%--------------------------------------------------------------------------
For a short interjection of one or two lines of text in the middle of a multiple-line display structure:
\begin{align}
A_1&=N_0(\lambda;\Omega')-\phi(\Omega)\\
A_2&=\phi(\lambda;\Omega')-\phi(\lambda)\\
\intertext{and}
A_3&=\mathcal{N}(\lambda;\omega).
\end{align}

\begin{verbatim}
\begin{align}
A_1&=N_0(\lambda;\Omega')-\phi(\Omega')\\
A_2&=\phi(\lambda;\Omega')-\phi(\lambda)\\
\intertext{and}
A_3&=\mathcal{N}(\lambda;\omega).
\end{align}
\end{verbatim}

\textbf{Note:} \verb!\intertext! may only appear right after a \verb!\\! or \verb!\\*! command.\\[6pt]

\noindent Also, see \Sec{ch:text-interjection-ieeeeqarray}.

%--------------------------------------------------------------------------
\noindent \dotfill
\section{\texttt{subequations}}
%--------------------------------------------------------------------------
This is working \verb!IEEEtran.sty! very well!\\
See \eqref{eq:101}, \eqref{eq:101-a}, and \eqref{eq:101-b}:
\begin{subequations} \label{eq:101}
	\begin{equation} \label{eq:101-a}
	\rho(r)=\sigma^2
	\end{equation}
	\begin{equation} \label{eq:101-b}
   	 M = \arctan\theta
	\end{equation}
\end{subequations}

\begin{verbatim}
\begin{subequations} \label{eq:}
	\begin{equation} \label{eq: -a}
	\rho(r)=\sigma^2 
	\end{equation}
	\begin{equation} \label{eq: -b}
	 M=\arctan\theta 
	\end{equation}
\end{subequations}
\end{verbatim}


%--------------------------------------------------------------------------
\subsection{\texttt{subequations} \& \texttt{align} together}
%--------------------------------------------------------------------------

%--------------------------------------------------------------------------
\subsubsection{Example~1}
%--------------------------------------------------------------------------
See \eqref{eq:124}, \eqref{eq:124-a}, \eqref{eq:124-b}, and \eqref{eq:124-c}:

\begin{subequations} \label{eq:124}
\begin{align}
\mat{A}&=\mat{B} \label{eq:124-a}\\
\mat{C}&=\mat{D} \label{eq:124-b}\\
\mat{E}&=\mat{F} \label{eq:124-c}\\
\end{align}
\end{subequations}


\begin{verbatim}
\begin{subequations} \label{eq:124}
	\begin{align}
	\mat{A}&=\mat{B} \label{eq:124-a}\\
	\mat{C}&=\mat{D} \label{eq:124-b}\\
	\mat{E}&=\mat{F} \label{eq:124-c}\\
	\end{align}
\end{subequations}
\end{verbatim}

%--------------------------------------------------------------------------
\subsubsection{Example~2 - MNA-PEEC}
%--------------------------------------------------------------------------
\begin{subequations} \label{eq:PEEC-1}
	\begin{align}
	\left(\mat{G}(\bs{\xi}) + s \mat{C}(\bs{\xi})\right)\mat X(s,\bs{\xi}) &=
	\mat{B}\mat{I}_p(s) \label{eq:PEEC-1a}\\
	\mat{V}_p(s,\bs{\xi})&=
	\mat{L}\mat{X}(s,\bs{\xi}) \label{eq:PEEC-1b}
	\end{align}
\end{subequations}

\begin{verbatim}
\begin{subequations} \label{eq:PEEC-1}
	\begin{align}
	\left(\mat{G}(\bs{\xi}) + s \mat{C}
	(\bs{\xi})\right)\mat X(s,\bs{\xi}) &=
	\mat{B}\mat{I}_p(s) \label{eq:PEEC-1a}\\
	\mat{V}_p(s,\bs{\xi})&=
	\mat{L}\mat{X}(s,\bs{\xi}) 
	\label{eq:PEEC-1b}
	\end{align}
\end{subequations}
\end{verbatim}

%--------------------------------------------------------------------------
\subsubsection{Example~3 - MNA}
%--------------------------------------------------------------------------
\begin{subequations}
	\begin{align}
	s\mathat{C}(s)\mathat{X}(s)+\mathat{G}\mathat{X}(s)&=
	\mathat{B}\mat{U}(s)\\
	\mat{I}_p(s)&=\mathat{B}\T\mathat{X}(s)
	\end{align}
	\text{where}
	\begin{align}
	\mathat{G}\trieq\mat{Q}\T\mat{G}\mat{Q}&&
	\mathat{C}\trieq\mat{Q}\T\mat{C}\mat{Q}&&
	\mathat{B}\trieq\mat{Q}\T\mat{B}
	\end{align}
\end{subequations}

\begin{verbatim}
\begin{subequations} \label{eq: }
\begin{align}
s\mathat{C}(s) \mathat{X}(s)+
\mathat{G}\mathat{X}(s)&=
\mathat{B}\mat{U}(s) \label{eq: -a}\\
\mat{I}_p(s)&=\mathat{B}\T
\mathat{X}(s) \label{eq: -b}
\end{align}
\text{where}
\begin{align}
\mathat{G}\trieq\mat{Q}\T\mat{G}\mat{Q}&&
\mathat{C}\trieq\mat{Q}\T\mat{C}\mat{Q}&&
\mathat{B}\trieq\mat{Q}\T\mat{B}
\end{align}
\end{subequations}
\end{verbatim}


%------------------------------------------------------------------------
\noindent \dotfill
\section{\texttt{gather}~{\color{red} \Large $\mathbf{\times}$}}
%------------------------------------------------------------------------
Equation groups without alignment.

%------------------------------------------------------------------------
\subsubsection{Example 1}~{\color{red} \Large $\mathbf{\times}$}
%------------------------------------------------------------------------

\begin{gather}
H_c = \\
\begin{split}
&= a^2\\
&= b^c\\
\end{split}\\
= H_d
\end{gather}

\begin{verbatim}
\begin{gather} \label{eq: }
H_c = \\
\begin{split}
&= a^2\\
&= b^c\\
\end{split} \\
= H_d
\end{gather}
%
\end{verbatim}


%------------------------------------------------------------------------
\subsubsection{Example 2}~{\color{red} \Large $\mathbf{\times}$}
%------------------------------------------------------------------------

\begin{gather*}
a_0=\frac{1}{\pi}\int\limits_{-\pi}^{\pi}f(x)\,\mathrm{d}x\\[6pt]
\begin{split}
a_n=\frac{1}{\pi}\int\limits_{-\pi}^{\pi}f(x)\cos nx\,\mathrm{d}x=\\
=\frac{1}{\pi}\int\limits_{-\pi}^{\pi}x^2\cos nx\,\mathrm{d}x
\end{split}\\[6pt]
\begin{split}
b_n=\frac{1}{\pi}\int\limits_{-\pi}^{\pi}f(x)\sin nx\,\mathrm{d}x=\\
=\frac{1}{\pi}\int\limits_{-\pi}^{\pi}x^2\sin nx\,\mathrm{d}x
\end{split}\\[6pt]
\end{gather*}

\begin{verbatim}
\begin{gather*}
a_0=\frac{1}{\pi}\int
\limits_{-\pi}^{\pi}f(x)\,\mathrm{d}x
\\[6pt]
\begin{split}
a_n=\frac{1}{\pi}\int\limits_{-\pi}
^{\pi}f(x)\cos nx\,\mathrm{d}x=\\
=\frac{1}{\pi}\int\limits_{-\pi}
^{\pi}x^2\cos nx\,\mathrm{d}x
\end{split}\\[6pt]
\begin{split}
b_n=\frac{1}{\pi}\int\limits_{-\pi}
^{\pi}f(x)\sin nx\,\mathrm{d}x=\\
=\frac{1}{\pi}\int\limits_{-\pi}^{\pi}x
^2\sin nx\,\mathrm{d}x
\end{split}\\[6pt]
\end{gather*}
\end{verbatim}

%------------------------------------------------------------------------
\noindent \dotfill
\section{\texttt{case} Package~{\color{red}\Large $\mathbf{\times}$}}
%------------------------------------------------------------------------
The \verb!cases! package provides \verb!numcases! and \verb!subnumcases! environment to produce multi-case equations with a separate equation number for each case. \\
Their difference is how they number the sub-equations:\\
\textbullet~ \texttt{numcases}: (1), (2), etc.\\
\textbullet~ \texttt{subnumcases}: (1a), (1b), etc. (hence the name subnum...)\\

%------------------------------------------------------------------------
\noindent \dotfill
\subsection{\texttt{numcases}~{\color{red}\Large $\mathbf{\times}$}}
%------------------------------------------------------------------------

\begin{verbatim}
\begin{numcases}{\label{t101} {\Sigma:}}
C\frac{d}{{dt}}x(t)+Gx(t)=Bu(t)
\label{eq: -a}\\
i(t)=Lx(t) \label{eq: -b}
\end{numcases}
\end{verbatim}
%
%
\begin{numcases}{\label{t101} {\Sigma:}}
C\frac{d}{{dt}}x(t)+Gx(t)=Bu(t)\\
i(t)=Lx(t)
\end{numcases}

Using \texttt{numcases} the numbers for equations are as (1) and (2) (they do not include letters 'a' and 'b').

%------------------------------------------------------------------------
\noindent \dotfill
\subsection{\texttt{subnumcases}~{\color{red} \Large $\mathbf{\times}$}}
%------------------------------------------------------------------------
\begin{subnumcases}{\boldsymbol{\Psi}:}
C\frac{d}{dt}x(t)+Gx(t)=Bu(t)\\
i(t)=Lx(t)
\end{subnumcases}

\begin{verbatim}
\begin{subnumcases}{\label{eq: } {\boldsymbol{\Psi}:}}
C\frac{d}{dt}x(t)+Gx(t)=Bu(t)
\label{eq: -a}\\
i(t)=Lx(t) \label{eq: -b}
\end{subnumcases}
\end{verbatim}

Note that in \texttt{subnumcases} environment the numbers for equations includes letters 'a' and 'b', hence the name, 'subnum'; e.g. (1a) and (1b). 

\newpage
%-------------------------------------------------------------------------------
\noindent \dotfill
\section{IEEEtran}
%-------------------------------------------------------------------------------
When using any general \verb!\documentclass{}! such as ``article, report, etc.'' to access the
IEEEtran facilities, include the following line in the preamble of your document:\\
\verb!\usepackage{IEEEtrantools}!

%-------------------------------------------------------------------------------
\subsection{IEEEproof}
%-------------------------------------------------------------------------------
\begin{IEEEproof}
This is a proof.
\end{IEEEproof}

\begin{verbatim}
\begin{IEEEproof}
This is a proof.
\end{IEEEproof}
\end{verbatim}
Manually access \verb!\hfill\IEEEQED! as \hfill\IEEEQED\\
Or by \verb!\hfill\IEEEQEDopen! as \hfill\IEEEQEDopen\\[4pt]

\noindent To change the default from closed to open (if needed):\\
\verb!\renewcommand{\IEEEQED}{\IEEEQEDopen}!

%-------------------------------------------------------------------------------
\par \noindent \dotfill
\section{\texttt{IEEEeqnarray}} \label{ch:ieeeeqarray}
%-------------------------------------------------------------------------------
IEEEtran also provides its own unique set of alignment tools which are known as the \verb!IEEEeqnarray! family. The IEEEeqnarray environment is for multiline equations
that occupy the entire column.

\begin{IEEEeqnarray}{rCl}
a & = & b + c \\
& = & d + e + f + g + h
+ i + j + k \nonumber\\
&& +\> l + m + n + o\\
& = & p + q + r + s
\end{IEEEeqnarray}

Compare using small ``c'' (below) with capital ``C'' (above).

\begin{IEEEeqnarray}{rcl}
a & = & b + c \\
  & = & d + e + f + g + h
	+ i + j + k \nonumber\\
     && +\> l + m + n + o\\
  & = & p + q + r + s
\end{IEEEeqnarray}


\begin{IEEEeqnarray}{c}
	a = \sum_{k=1}^n\sum_{\ell=1}^n
	\sin \bigl(2\pi \, e_{\ell} \, h \bigr)
	\IEEEeqnarraynumspace
	\label{eq:labelc1}
\end{IEEEeqnarray}


%-------------------------------------------------------------------------------
\par \noindent \dotfill
\subsection{Interjecting text in \texttt{\textbackslash IEEEeqbarray}}
\label{ch:text-interjection-ieeeeqarray}
%-------------------------------------------------------------------------------
\verb!\noalign{}! command can be used (within IEEEeqnarray family) to inject text which is outside of the alignment structure.

\begin{IEEEeqnarray}{rCl}
	\IEEEyesnumber
	A_1&=&7  \IEEEyessubnumber\\
	A_2&=&b+1\IEEEyessubnumber\\
	\noalign{\noindent and\vspace{\jot}}
	A_3&=&d+2\IEEEyessubnumber%
\end{IEEEeqnarray}

\begin{verbatim}
\begin{IEEEeqnarray}{rCl}
	\IEEEyesnumber
	A_1&=&7  \IEEEyessubnumber\\
	A_2&=&b+1\IEEEyessubnumber\\
	\noalign{\noindent and\vspace{\jot}}
	A_3&=&d+2\IEEEyessubnumber%
\end{IEEEeqnarray}
\end{verbatim}

Remember the issues associated with using \verb!\noalign{}!:
\begin{enumerate}
\item \verb!\noalign! places its contents outside of the alignment;
\item It does not automatically place its contents within a box;
\item Issues with page-breaks occur around \verb!\noalign!.
\end{enumerate}

Also, see \Sec{ch:text-interjection}.


%-------------------------------------------------------------------------------
\par \noindent \dotfill
\subsection{\texttt{\textbackslash IEEEeqbarray} Cases Structures}
%-------------------------------------------------------------------------------
\begin{equation}
\setlength{\nulldelimiterspace}{0pt}
|x|=\left\{
\begin{IEEEeqnarraybox}[\relax][c]{l's}
 x,&for $x \geq 0$\\
-x,&for $x < 0$%
\end{IEEEeqnarraybox}\right.
\end{equation}


\begin{verbatim}
\begin{equation}
\setlength{\nulldelimiterspace}{0pt}
|x|=\left\{
\begin{IEEEeqnarraybox}[\relax][c]{l's}
x,&for $x \geq 0$\\
-x,&for $x < 0$%
\end{IEEEeqnarraybox}\right.
\end{equation}
\end{verbatim}

%-------------------------------------------------------------------------------
\par \noindent \dotfill
\subsection{Matrices in \texttt{\textbackslash IEEEeqbarray}}
\label{ch:mat-in-ieeeeqarray}
%-------------------------------------------------------------------------------

\begin{equation}
I = \left(\begin{IEEEeqnarraybox*}[][c]{,c/c/c,}
1&0&0\\
0&1&0\\
0&0&1%
\end{IEEEeqnarraybox*}\right)
\end{equation}

Also, see \Sec{ch:mat-in-amsmath}.
\newpage
%-------------------------------------------------------------------------------
\par \noindent \dotfill
\section{Matrices in \texttt{amsmath}} \label{ch:mat-in-amsmath}
%-------------------------------------------------------------------------------
\begin{equation}
I = \left(\begin{IEEEeqnarraybox*}[][c]{,c/c/c,}
1&0&0\\
0&1&0\\
0&0&1%
\end{IEEEeqnarraybox*}\right)
\end{equation}

\begin{verbatim}
\begin{equation}
I = \left(\begin{IEEEeqnarraybox*}[][c]{,c/c/c,}
1&0&0\\
0&1&0\\
0&0&1%
\end{IEEEeqnarraybox*}\right)
\end{equation}
\end{verbatim}



%-------------------------------------------------------------------------------
\subsection{\texttt{\textbackslash bmatrix}}
%-------------------------------------------------------------------------------
\[\begin{bmatrix} 
A & B & C \\ 
D & E & F \\ 
G & H & I 
\end{bmatrix}\]

\begin{verbatim}
\begin{bmatrix} 
A & B & C \\ 
D & E & F \\ 
G & H & I 
\end{bmatrix}
\end{verbatim}

$$
\begin{bmatrix}
-\frac{\partial G}{\partial \lambda} &
-\frac{\partial C}{\partial \lambda}
\end{bmatrix}
\begin{bmatrix}	
x(t) \\
\frac{dx(t)}{dt} 
\end{bmatrix}
$$

\begin{verbatim}
\begin{bmatrix}
-\frac{\partial G}{\partial \lambda} &
-\frac{\partial C}{\partial \lambda}
\end{bmatrix}
\begin{bmatrix}	
x(t) \\
\frac{dx(t)}{dt} 
\end{bmatrix}
\end{verbatim}

%-------------------------------------------------------------------------------
\subsection{\texttt{\textbackslash array}}
%-------------------------------------------------------------------------------

$$\left[\begin{array}{ccc}
A & B & C \\ 
D & E & F \\ 
G & H & I 
\end{array}\right]$$

\begin{verbatim}
\left[\begin{array}{ccc}
A & B & C \\ 
D & E & F \\ 
G & H & I 
\end{array}\right]
\end{verbatim}

$$\left[\begin{array}{cc}
-\frac{\partial G}{\partial \lambda} &
-\frac{\partial C}{\partial \lambda}
\end{array}\right]
\left[\begin{array}{c}	
x(t) \\
\frac{dx(t)}{dt} 
\end{array}\right]$$


\begin{verbatim} 
\left[\begin{array}{cc}
-\frac{\partial G}{\partial \lambda} &
-\frac{\partial C}{\partial \lambda}
\end{array}\right]
\left[\begin{array}{c}	
x(t) \\
\frac{dx(t)}{dt} 
\end{array}\right]
\end{verbatim}

%------------------------------------------------------------------------
\noindent \dotfill
\subsubsection{More with \texttt{\textbackslash array}}
%------------------------------------------------------------------------

$$ \left[ \begin{array}{c|c|c}
1 & 2 & 3 \\ \hline
4 & 5 & 6 \\ \hline
7 & 8 & 9
\end{array} \right] $$

\begin{center}
	\begin{verbatim}
	\left[ \begin{array}{c|c|c}
	1 & 2 & 3 \\ \hline
	4 & 5 & 6 \\ \hline
	7 & 8 & 9
	\end{array} \right] 
	\end{verbatim}
\end{center}


$$ \begin{array}{r|rrr}
& a & b & c \\ \hline
1 & 2 & 3 & 4 \\
2 & 3 & 4 & 5 \\
3 & 4 & 5 & 0
\end{array} $$


\begin{verbatim}
\begin{array}{r|rrr}
& a & b & c \\ \hline
1 & 2 & 3 & 4 \\
2 & 3 & 4 & 5 \\
3 & 4 & 5 & 0
\end{array}
\end{verbatim}

\noindent \dotfill

$$\left[\begin{array}{ccc} 
1, & \ldots, & 3 
\end{array}\right] $$
$$\left[\begin{array}{*{20}{c}} 
1, & \ldots, & 3 
\end{array}\right] $$

\begin{verbatim}
\left[\begin{array}{*{20}{c}} 
1, & \ldots, & 3 
\end{array}\right]
\end{center}
\end{verbatim}


%------------------------------------------------------------------------
\noindent \dotfill
\subsection{\textbackslash pmatrix}
%------------------------------------------------------------------------
To create inline small-matrix.\\[6pt]

\noindent This is an example 
$\begin{pmatrix}
a + b + c & uv\\
a + b & c + d
\end{pmatrix}$ 
showing small matrix.

\begin{verbatim}
$\begin{pmatrix}
a + b + c & uv\\
a + b & c + d
\end{pmatrix}$
\end{verbatim}



%------------------------------------------------------------------------
\noindent \dotfill
\subsection{\textbackslash smallmatrix}
%------------------------------------------------------------------------
This is an example $\left(
\begin{smallmatrix}
a + b + c & uv\\
a + b & c + d
\end{smallmatrix}
\right)$ showing smaller matrix.


\begin{verbatim}
$\left(
\begin{smallmatrix}
a + b + c & uv\\ 
a + b & c + d    
\end{smallmatrix} 
\right)$ 
\end{verbatim}


Also, see \ref{ch:mat-in-ieeeeqarray}.\\


\noindent \dotfill

%--------------------------------------------------------------------------
%\bibliography{\ieeebib}
%--------------------------------------------------------------------------
\end{document}


